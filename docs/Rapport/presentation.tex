\documentclass{beamer}
 
\usepackage[english]{babel}
\usepackage[utf8]{inputenc}
\usepackage[T1]{fontenc}
\usepackage{lmodern}
\usepackage{graphicx}
\usepackage{hyperref}
\hypersetup{colorlinks=true,urlcolor=blue,linkcolor=white}

						 
\usetheme{Warsaw}

\begin{document}

	\title{Sudoku}
	\subtitle{Présentation du développement du projet}
	\author{Équipe Sudoku}	
	\institute{Université Montpellier II}
	\date{}

	\maketitle

	% MEMBRES DE L'ÉQUIPE %

	\begin{frame}
	\frametitle{Équipe}
		\begin{centering}
			\href{mailto:abdoulaye.diallo@etud.univ-montp2.fr}{Abdoulaye Diallo}\\ 			\href{mailto:redoine.el-ouasti@etud.univ-montp2.fr}{Redoine El Ouasti}\\
			Assistant chef de projet : \href{mailto:simon.galand@etud.univ-montp2.fr}{Simon Galand}\\
			Assistant chef de projet : \href{mailto:adrien.lamant@etud.univ-montp2.fr}{Adrien Lamant}\\
			\href{mailto:pierre-louis.latour@etud.univ-montp2.fr}{Pierre-Louis Latour}\\
			\href{mailto:charly.maeder@etud.univ-montp2.fr}{Charly Maeder}\\
			\href{mailto:pierre.ruffin@etud.univ-montp2.fr}{Pierre Ruffin}\\ 
			Chef de projet : \href{mailto:stella.zevio@etud.univ-montp2.fr}{Stella Zevio}\\
		\end{centering}
	\end{frame}

	% SOMMAIRE %
	
	\begin{frame}
	\frametitle{Sommaire}
		\tableofcontents
		   \begin{itemize}
		   \item Introduction
		   \item Conception \& Méthodes
		   \item Développement
		   \item Conclusion
   		\end {itemize}
	\end{frame}


	% INTRODUCTION %

	\begin{frame}
	\frametitle{Introduction}
	\begin{itemize}
		\item Génération de grilles de sudoku de taille 4*4, 9*9, 16*16
		\item Expérience de jeu
	\end {itemize}

	\begin{figure}[H]
		\centering
		\includegraphics[width=0.4\textwidth]{images/grille}
	\end{figure}
	\end{frame}

	% CONCEPTION ET METHODES %

	\begin{frame}
	\frametitle{Conception \& Méthodes}

	\begin{itemize}
	\item Analyse du sujet
		\begin{itemize}
			\item Cahier des charges
			\begin{itemize}
				\item Fonctions à implémenter
				\item Organisation de l'équipe
			\end{itemize}
			\item Diagramme de Gantt
		\end{itemize}
	\item Outils utilisés
		\begin{itemize}
			\item Éditeur de texte (Gedit, Emacs) ou IDE (CodeBlocks, Xcode)
			\item Langage C
			\item gcc
			\item Valgrind
			\item GTK+
			\item Doxygen 
			\item LaTeX
		\end{itemize}
	\end{itemize}
	 	
	\end{frame}

	% DEVELOPPEMENT %

	\begin{frame}
	\frametitle{Développement}
	\framesubtitle{Fonctions implémentées}
		\begin{itemize}
			\item Génération des grilles complètes (solution)
			\item Génération des grilles de jeu
			\item Jeu
			\item Aide
			\item Sauvegarde
			\item Chargement
			\item Réinitialisation
			\item Validation
			\item Classement
			\item Solveur
		\end{itemize}

	\end{frame}

	% BACKTRACKING %
	
	\begin{frame}
	\frametitle{Retour sur trace (\textit{backtracking})}
		\begin{definition}
		Algorithme consistant à revenir en arrière sur les décisions prises pour sortir d'un blocage (ici revenir à un état de la grille valide lors d'une erreur de remplissage).
		\end{definition}
	\end{frame}
	
	% HEURISTIQUES %

	\begin{frame}
	\frametitle{Heuristiques}
	\begin{definition}
        Méthode de calcul en optimisation combinatoire fournissant rapidement une solution réalisable, pas nécessairement optimale.
   	\end{definition}
   	\begin{description}
        	\item[Cross-hatching] supprimer des listes de possibilités d’une sous-grille les valeurs qui sont déjà certaines
		\item[Lone-number] fixer une valeur à partir du moment où elle n'apparaît que dans une seule liste de possibilités de la sous-grille
		\item[Naked subset] retirer de certaines listes de possibilités des valeurs appartenant à des N-uplets apparaissant N fois dans d'autres listes (N étant un entier non nul inférieur à la taille de la grille)
	\end{description}
	\end{frame}

		
	% CONCLUSION %

	\begin{frame}
	\frametitle{Conclusion}
	\begin{itemize}
    		\item Outils choisis
    		\item Choix du sujet
    		\item Problèmes rencontrés
    		\item Harmonie du groupe
    		\item Compétences acquises
    		\item Ressentis
    	\end{itemize}
 	\end{frame}

\end{document}
